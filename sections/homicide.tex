\section{Homicide}

\subsection{Intentional Killing}

\subsubsection{Murder}

\paragraph{\emph{State v. Guthrie}}

\begin{enumerate}
    \item The defendant stabbed and killed a coworker after the coworker taunted him and snapped him in the nose with a towel. The trial court found him guilty of first degree murder. The defendant argues that the trial court's instructions to the jury were improper because ``the terms wilful, deliberate, and premeditated were equated with a mere intent to kill.'' The appellate court agreed that ``premeditation'' cannot be synonymous with intent---rather, it must be long enough for the defendant to be ``fully conscious of what he intended.'' Reversed and remanded for a new trial.
\end{enumerate}

\paragraph{\emph{Midgett v. State}}

\begin{enumerate}
    \item The defendant repeatedly abused his young son, who died from the injuries. The trial court found him guilty of first degree murder, which required premeditation and deliberation. The defendant argues that there was no premeditation, and the Supreme Court of Arkansas agreed. The dissent argued that symptoms of malnourishment indicated starvation, but the majority argued that the evidence did not prove starvation.
    \item Shortly after this case, the Arkansas legislature amended its criminal code to broaden first degree murder to include ``extreme indifference to the value of human life'' of people fourteen years old or younger.
\end{enumerate}

\paragraph{\emph{State v. Forrest}}

\begin{enumerate}
    \item The defendant shot and killed his terminally ill father in the hospital. The trial court convicted him of first degree murder. The defendant argued that there was no premeditation or deliberation. The appellate court agreed with the trial court, noting that premeditation must be proved by circumstantial evidence, including provocation from the victim, the defendant's conduct and statements, ill will between the parties, lethal blows after the victim was rendered helpless, and evidence of an especially brutal killing. In this case, the court found that the victim was lying helpless and did nothing to provoke the defendant, and that the defendant had earlier made statements about ``putting his father out of his misery.''
\end{enumerate}

\subsubsection{Manslaughter}

\paragraph{\emph{Girouard v. State}}

\begin{enumerate}
    \item Are words enough to satisfy the provocation requirement for reducing murder to manslaughter?
    \item The defendant stabbed and killed his wife after she taunted him relentlessly. The trial court convicted him of second degree murder. He argued on appeal that the rule of provocation should be expanded to include verbal provocation. The appellate court relied on the rule that for provocation to mitigate a charge of murder, it must be ``calculated to inflame the passion of a reasonable man and tend to cause him to act for the moment from passion to reason.'' The standard is objective. The court found that words can constitute adequate provocation if they accompany intent and ability to cause bodily harm. That was not the case in this scenario, however. The court upheld the second degree murder conviction.
\end{enumerate}

\paragraph{\emph{People v. Casassa}}

\begin{enumerate}
    \item The defendant stabbed and killed his neighbor out of jealousy. The trial court found him guilty of second degree murder. The defendant argued he was acting under ``extreme emotional disturbance,'' which would reduce the charge to manslaughter. The appellate court reasoned that the emotional disturbance must meet an objectively reasonable standard. In this case, the disturbance was a result of the defendant's unique mental state---i.e., a reasonable personal would not have been so emotionally disturbed under the circumstances.
\end{enumerate}

\subsection{Unintentional Killing}

\paragraph{\emph{People v. Knoller}}

\paragraph{\emph{State v. Hernandez}}

\paragraph{\emph{State v. Williams}}

