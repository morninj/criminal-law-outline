\section{Elements of a Crime}

\begin{enumerate}
    \item Every crime has two elements: \emph{actus reus} and \emph{mens rea}.
    \item Every crime also has attendant circumstances.
\end{enumerate}

\subsection{Actus Reus}

\begin{enumerate}
    \item Literally, ``guilty act.'' There is no universally accepted definition. In murder, for instance, some would consider it to be the pulling of the trigger. Others would consider it to be the death itself. The most common definition would consider it to be both.
    \item ``Omissions are not accidents.''---Marianne Moore
    \item What constitutes an act? When does the act begin? See Model Penal Code § 2.01.
    \item If someone holds a gun to your head and tells you to act, your act is voluntary. An act is something you do wilfully.
    \item Thought crimes are not punishable (\emph{Minority Report}, \emph{Firestarter}).
\end{enumerate}

\subsubsection{\emph{Martin v. State}}

\begin{enumerate}
    \item Police officers took a drunk man from his home and onto a public highway, where they then arrested him for public drunkenness. The court held that public drunkenness cannot be established when the accused was involuntarily carried to a public place.
\end{enumerate}

\subsubsection{\emph{State v. Utter}}

\begin{enumerate}
    \item Defendant (here, the appellant) was drunk and stabbed his son. He had no memory of the stabbing. He argued that his service in the army had caused him to develop a ``conditioned response'' in which he reacts violently and involuntarily to people approaching unexpectedly from behind. The court reasons that an ``act'' requires voluntary action---that is, ``act'' is synonymous with ``voluntary act.'' An involuntary or unconscious act cannot induce guilt---that is, it is not an ``act'' at all. The court finds that the defendant's theory of conditioned response should have been presented to a jury \emph{if there was substantial evidence to support it.} However, because the jury could not possibly know or infer what had happened in the room at the time of the stabbing, the question should not be sent to the jury.
\end{enumerate}

\subsubsection{\emph{People v. Beardsley}}

\begin{enumerate}
    \item While his wife was away, the defendant was drinking heavily with a woman at his house. The woman took several tablets of morphine and became unresponsive. The defendant put her in a basement room in his house (which another man was renting). The woman died that evening. The issue is whether the defendant had a legal duty to protect the woman. If he omitted to perform his duty, he would be criminally liable for manslaughter. The prosecution argued that the defendant was in the role of the woman's guardian. The court reasoned, however, that if the defendant had been drinking with a man and that man attempted suicide, the defendant would not have had a duty to protect him---so it should make no difference that he was with a woman.
\end{enumerate}

\subsubsection{\emph{Barber v. Superior Court}}

\begin{enumerate}
    \item A patient suffered cardiac arrest after surgery. Doctors managed to save him, but he suffered significant brain damage. He remained in vegetative state on life support with little chance of recovery. His family decided to remove him from life support, and he died a few days later. The question is whether his doctors had a duty to keep him alive---since omitting to perform that duty would make them liable for murder. \textbf{``There is no criminal liability for failure to act unless there is a legal duty to act.''} The court reasons first that removing the man from life support constituted an omission, not a positive act. The court holds that the decision of whether to continue treatment was left to the family. Therefore, the doctors did not unlawfully fail to perform a legal duty.
\end{enumerate}

\subsubsection{\emph{Lawrence v. Texas}}

\begin{enumerate}
    \item In an opinion from Justice Kennedy, the court decided whether a Texas law criminalizing sodomy violates the Fourteenth Amendment's Due Process Clause and equal protection guarantee. It held that the statute violates individuals' rights to privacy and liberty. It overturned an earlier ruling on a similar Georgia statute in \emph{Bowers v. Hardwick}: ``\emph{Bowers} was not correct when it was decided, and it is not correct today. It ought not to remain binding precedent.''
\end{enumerate}

\subsection{Mens Rea}

\begin{enumerate}
    \item ``Guilty mind.''
    \item \emph{United States v. Cordoba-Hincapie} includes a brief history of the evolution from ancient English strict liability to modern requirement of a guilty state of mind.
    \item Holmes has two definitions:
    \begin{enumerate}
        \item A morally culpable state of mind in general.
        \item The mental state specified in the definition of the crime.
    \end{enumerate}
\end{enumerate}

\subsubsection{\emph{Regina v. Cunningham}}

\begin{enumerate}
    \item The defendant stole a gas meter from the basement of his mother-in-law's house, causing noxious gas to escape and partially asphyxiate his neighbor. The issue was whether his action was malicious. A lower court convicted the defendant on the definition of malice as ``wickedness,'' i.e., a generally culpable state of mind. This court defined malice as (1) an \textbf{intention} to do the specific harm, or (2) \textbf{recklessness} (i.e., he foresaw that the harm \emph{might} occur, but did it anyway). The court overturned the conviction.
\end{enumerate}

\subsubsection{\emph{People v. Conley}}

\begin{enumerate}
    \item In a fight after a high school party, the defendant smashed a wine bottle into the victim's face, causing permanent disability. He intended to hit someone else (who ducked), but the court found that the defendant nonetheless intended his action to cause permanent disability.
    \item The common law definition of intent includes both the actor's conscious goal and the results that are ``virtually certain to occur.''\footnote{Casebook p. 155.} Analogous to substantial certainty in intentional torts.
    \item A person ``intends the natural and probable consequences of his actions.'' The Fourteenth Amendment prevents courts from presuming this, but juries can use common sense to recognize it.
    \item \textbf{Transferred intent} (in criminal law) allows transfer from one victim to another. Transfer between different types of harms is less clear cut. Courts often apply it, but not always.
    \item There is dispute about the meaning of ``general intent'' and ``specific intent.'' Some versions include:
    \begin{enumerate}
        \item General: the definition of the crime sets out no specific mental state, so the prosecutor needs only to prove a generally culpable state of mind. Specific: the definition of a crime explicitly sets out a mental state.
        \item General: reserved for crimes that permit conviction on the basis of a less culpable mental state (e.g., negligence or recklessness). Specific: denotes an offense that includes a definition of intent or knowledge.
        \item General: any mental state related to the act that constitutes the social harm. Specific: an additional ``special mental element'': (1) intent to commit a future act (e.g., intent to sell), (2) special motive (e.g., offensive contact intended to cause humiliation), or (3) knowledge of attendant circumstances (e.g., sale of obscene material to a minor).
    \end{enumerate}
\end{enumerate}

\subsubsection{MPC § 2.02: General Requirements of Culpability}

\begin{enumerate}
    \item Requires ``elemental'' culpability---i.e., the specific state of mind required in the definition of the crime, rather than a generally morally culpable state of mind.
    \item There are four levels of culpability:
    \begin{enumerate}
        \item \textbf{Purpose}: An actor intends to perform a specific action or to cause a specific result.
        \item \textbf{Knowledge}: An actor is aware of factual circumstances that establish criminal culpability. For instance, someone may not \emph{intend} to commit statutory rape, but knowledge that the victim is younger than the age of consent is enough to establish guilt.
        \item \textbf{Recklessness}: An actor creates and recognizes a substantial, unjustifiable risk and acts anyway. The jury should decide whether the risk is substantial and unjustifiable and whether \emph{disregard of the risk} deserves condemnation. 
        \item \textbf{Negligence}: An actor inadvertently creates a substantial, unjustifiable risk of which he should have been aware. The jury should decide whether the risk is substantial and unjustifiable and whether the defendant's \emph{failure to perceive the risk} deserves condemnation.
    \end{enumerate}
    \item If a law does not specify a culpable state of mind, culpability is established if the person acted purposefully, knowingly, or recklessly. Negligence is excluded unless the law specifically prescribes it.
\end{enumerate}

\subsubsection{Knowledge of Attendant Circumstances}

\begin{enumerate}
    \item todo.
\end{enumerate}

\subsubsection{\emph{State v. Nations}}

\begin{enumerate}
    \item todo.
\end{enumerate}

\subsection{Strict Liability}

\begin{enumerate}
    \item todo.
\end{enumerate}

\subsubsection{\emph{United States v. Cordoba-Hincapie}}

\begin{enumerate}
    \item todo.
\end{enumerate}

\subsubsection{\emph{Staples v. United States}}

\begin{enumerate}
    \item todo.
\end{enumerate}

\subsubsection{\emph{Garnett v. State}}

\begin{enumerate}
    \item todo.
\end{enumerate}

\subsection{Mistake and Mens Rea}

\begin{enumerate}
    \item todo.
\end{enumerate}

\subsubsection{Mistake of Fact: \emph{People v. Navarro}}

\begin{enumerate}
    \item todo.
\end{enumerate}

\subsubsection{Mistake of Law: \emph{People v. Navarro}}

\begin{enumerate}
    \item todo.
\end{enumerate}

\subsubsection{\emph{Cheek v. United States}}

\begin{enumerate}
    \item todo.
\end{enumerate}
