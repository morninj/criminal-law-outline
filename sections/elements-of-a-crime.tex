\section{Elements of a Crime}

Every crime has two elements: \emph{actus reus} and \emph{mens rea}.

\subsection{Actus Reus}

\begin{enumerate}
    \item Literally, ``guilty act.'' There is no universally accepted definition. In murder, for instance, some would consider it to be the pulling of the trigger. Others would consider it to be the death itself. The most common definition would consider it to be both.
\end{enumerate}

\subsubsection{\emph{Martin v. State}}

\begin{enumerate}
    \item Police officers took a drunk man from his home and onto a public highway, where they then arrested him for public drunkenness. The court held that public drunkenness cannot be established when the accused was involuntarily carried to a public place.
\end{enumerate}

\subsubsection{\emph{State v. Utter}}

\begin{enumerate}
    \item Defendant (here, the appellant) was drunk and stabbed his son. He had no memory of the stabbing. He argued that his service in the army had caused him to develop a ``conditioned response'' in which he reacts violently and involuntarily to people approaching unexpectedly from behind. The court reasons that an ``act'' requires voluntary action---that is, ``act'' is synonymous with ``voluntary act.'' An involuntary or unconscious act cannot induce guilt---that is, it is not an ``act'' at all. The court finds that the defendant's theory of conditioned response should have been presented to a jury \emph{if there was substantial evidence to support it.} However, because the jury could not possibly know or infer what had happened in the room at the time of the stabbing, the question should not be sent to the jury.
\end{enumerate}
