\section{Basic Principles of Criminal Law}

\subsection{Introduction}

\begin{enumerate}
    \item Henry Hart argues that criminal law is a method with five features:
    \begin{enumerate}
        \item It operates by a series of commands (``don't kill or steal'').
        \item A community makes the commands binding.
        \item There are sanctions for disobeying the commands.
        \item The distinction between civil and criminal sanctions is that 
        criminal violations draw a community's moral condemnation.
        \item Violations are punished.
    \end{enumerate}
    \item Murray: laws are framed as conditions (``if you do x, then 
    y''---e.g., punishment), emphasizing \textbf{agency and choice}.
    \item \textbf{The legality principle}: \emph{Nullem crimen sine lege, 
    nulla poene sine lege} (``no crime 
    without law, no punishment without law'').
    \item Sources of criminal law:
    \begin{enumerate}
        \item Codification (statutes, administrative rules, etc.).
        \item Common law (based on the English system, as distinct from a 
        civil-law system).
        \item Case law.
        \item Model Penal Code.
    \end{enumerate}
    \item Themes in criminal law;
    \begin{enumerate}
        \item What is a crime? What is criminal law for?
        \item What distinguishes criminal conduct?
        \item Why do we use criminal law to address wrongs rather than another 
        system (torts, contracts)?
        \item How do we punish crime? What are the limits on punishment?
    \end{enumerate}
    \item What distinguishes criminal punishment?
    \begin{enumerate}
        \item Criminal penalties can restrain personal liberty (but civil 
        penalties don't).
        \item Moral stigma.
        \item Judgment is collective---it isn't about two 
        parties.\footnote{See Schelling, ``Ethics, Law, and the Exercise of 
        Self-Command.''}
    \end{enumerate}
    \item \textbf{Probable cause} is necessary to make an arrest.
    \item \textbf{Indictment} by a grand jury is usually necessary before a 
    case can go to trial.
    \item The Sixth Amendment guarantees a right to a speedy and public trial
    by an impartial jury.
    \item \textbf{Due process clauses} in the Fifth and Fourteenth amendments 
    guarantee persuasion \textbf{beyond a reasonable doubt} (as determined by 
    the factfinder---either a judge or jury).
    \item The Tenth Amendment reserves unenumerated powers for the states.
\end{enumerate}

\subsubsection{Beyond a Reasonable Doubt: \emph{Owens v. State}}

Circumstantial evidence can meet the reasonable doubt standard. The definition 
of ``beyond a reasonable doubt'' is contested and jury instructions vary.

\begin{enumerate}
    \item Owens was found drunk and asleep behind the wheel of a running car 
    in a private driveway.
    \item Circumstantial evidence gave equal weight to two interpretations of 
    the facts: either he had just arrived (which would make him guilty of 
    driving while intoxicated) or had not yet left (not guilty).
    \item If each interpretation is equally likely, the factfinder could not 
    fairly choose the guilty option beyond a reasonable doubt. But after 
    analyzing the evidence, the court found ``the totality of the 
    circumstances are, in the last analysis, inconsistent with a reasonable 
    hypothesis of innocence.''\footnote{Casebook p. 17.}
    \item The trial court convicted him of driving while intoxicated. The 
    appellate court affirmed.
\end{enumerate}

\subsection{Principles of Punishment}

\begin{enumerate}
    \item Some types of punishment: prison, fines, community service, shaming.
    \item Two key questions:
    \begin{enumerate}
        \item Who should be punished?
        \item How much punishment is appropriate?
    \end{enumerate}
    \item There are two predominant (and non-mutually-exclusive) theories of 
    punishment: \textbf{retributivism} and \textbf{utilitarianism}.
\end{enumerate}

\subsubsection{Utilitarianism}

Punishment is justified because it's useful.

\begin{enumerate}
    \item Jeremy Bentham: the \textbf{principle of utility} evaluates actions 
    in light of their effect on the happiness of the interested party. Laws 
    aim to augment a community's total happiness.
    \item Kent Greenawalt: ``Since punishment involves pain, it can be 
    justified only if it accomplishes enough good consequences to outweigh 
    this harm.''\footnote{Casebook p. 35.} The consequences of an action 
    determine its morality (or, the ends justify the means).
    \item Goals of utilitarian punishment:
    \begin{enumerate}
        \item General deterrence (i.e., discourage an action from occurring in 
        general within a community).
        \item Specific deterrence (i.e., discourage a specific person from 
        doing something harmful).
        \item Incapacitation.
        \item Reform.
    \end{enumerate}
\end{enumerate}

\subsubsection{Retributivism}

Punishment is justified because criminals deserve it.

\begin{enumerate}
    \item Michael Moore: ``the desert of an offender is a sufficient reason to 
    punish him or her.''\footnote{Casebook p. 39}
    \item Immanuel Kant: penal law is a categorical imperative.
    \item James Fitzjames Stephen: \textbf{assaultive retribution} holds that 
    hatred and vengeance in the name of morality are socially beneficial. 
    Punishment expresses our collective hatred. Criminals are ``noxious 
    insects.''\footnote{Casebook p. 42.}
    \item Herbert Morris: \textbf{protective retribution} holds that rules 
    exist to provide collective benefit. They guards against unfair advantage 
    for freeriders. If somebody cheats, punishment evens the score.
    \item Jeffrey G. Murphy \& Jean Hampton: \textbf{victim vindication 
    retribution} wrongdoers implicitly place their own value above their 
    victims'; ``retributive punishment is the defeat of the wrongdoer at the 
    hands of the victim.''\footnote{Casebook p. 46.}
\end{enumerate}

\subsubsection{Justifying Punishment}

\paragraph{Retributive and Utilitarian Justifications for Punishment: 
\emph{The Queen v. Dudley and Stephens}}
~\\\\
Utilitarian and retributivist justifications for punishment can lead to 
divergent results.

\begin{enumerate}
    \item Dudley, Stephens, Brooks, and Parker were castaways on a boat 1600 
    miles from the Cape of Good Hope. They quickly ran out of food and water. 
    After twenty days, Dudley and Stephens decided to kill and eat Parker 
    (with Brooks dissenting). All of them ate Parker's body for four days, at 
    which point they were rescued and brought to trial.
    \item The court found them guilty of murder. They were pardoned soon 
    after.
    \item This case highlights the differences between retributive and 
    utilitarian theories of justice. Parker was weak and unlikely to have 
    survived the last four days before rescue arrived. Dudley and Stephens 
    likely wouldn't have survived, either. Moreover, Dudley and Stephens had 
    family responsibilities, while Parker was a drifter.
    \item A retributive response would find that Dudley and Stephens are 
    morally culpable and should be found guilty regardless of the mitigating 
    factors.
    \item A utilitarian response would find them not guilty on the recognition 
    of a net benefit for all parties involved (except Parker, but he would 
    have died regardless). However, a utilitarian might want to deter 
    castaways from eating each other in case they end up being rescued.
\end{enumerate}

\paragraph{Sentencing: \emph{People v. Du}}
~\\\\
Courts can draw on both utilitarian and retributivist rationals in sentencing.

\begin{enumerate}
    \item The defendant, Soon Ja Du, a 51-year-old woman, owned a liquor store 
    in LA. A 15-year-old girl, Latasha Harlins, in the store put a bottle of 
    orange juice in her backpack. It's not clear whether she intended to pay. 
    A fight ensued, in which Du was injured. As Harlins was leaving, Du pulled 
    out a gun (which had been previously stolen, modified to fire on a hair 
    trigger, and then recovered) and shot Harlins in the back of the head.
    \item Du testified that she did not intend to kill Harlins. The jury 
    rejected her defense and convicted her of voluntary manslaughter.
    \item Du's probation officer concluded she ``would be most unlikely to 
    repeat this or any other crime.'' The sentencing court sentenced Du to ten 
    years, but suspended the sentence and placed her on probation. The 
    probation officer wrote, ``it is my opinion that justice is never served 
    when public opinion, prejudice, revenge or unwarranted sympathy are 
    considered by a sentencing court in resolving a case.'' She tested Du's 
    case against seven goals of sentencing:
    \begin{enumerate}
        \item Protect society.
        \item Punish the defendant.
        \item Encourage the defendant to lead a law-abiding life.
        \item Deter others.
        \item Incapacitation.
        \item Secure restitution for the victim.
        \item Seek uniformity in sentencing.
    \end{enumerate}
    \item None of these reasons were sufficient to justify prison time. The 
    only somewhat convincing motivation for prison time was the strong 
    presumption against probation when guns are involved. But this was an 
    unusual case, she concluded, ``which overcomes the statutory presumption 
    against probation.''
\end{enumerate}

\subsection{Proportionality of Punishment}

\subsubsection{General Principles}

\begin{enumerate}
    \item Kant: The ``right of retaliation'' (\emph{jus talionis}) is ``the 
    only principle which in regulating a public court...can definitely assign 
    both the quality and the quantity of a just penalty.''\footnote{Casebook 
    p. 70.} Murderers must be punished with death.
    \item Bentham: punishment has four goals:
    \begin{enumerate}
        \item General deterrence.
        \item Encourage criminals to choose the lesser of two offenses.
        \item Encourage criminals to do no more mischief than necessary.
        \item Punish cheaply.
    \end{enumerate}
    \item ...and five rules:
    \begin{enumerate}
        \item To effectively deter, the value of the punishment must be 
        greater than the value of the offense.
        \item The greater the mischief, the greater the punishment.
        \item Punishment must be sufficient to induce criminals to choose the 
        lesser of two crimes.
        \item Punishment must be adapted to each offense.
        \item Punishment should not be greater than necessary.
    \end{enumerate}
    \item The \emph{Eighth Amendment} prohibits disproportionate and cruel and 
    unusual punishment. MPC \S\ 1.02(2)(c) is in accord.
\end{enumerate}

\subsubsection{Constitutional Principles}

\paragraph{Rape and Capital Punishment: \emph{Coker v. Georgia}}
~\\\\
Rape does not involve the taking of life. The death penalty is therefore a 
disproportionate punishment in violation of the Eighth Amendment.

\begin{enumerate}
    \item The defendant escaped from prison, where he was serving time for 
    multiple violent felonies. He broke into the Carvers' house, tied up Mr. 
    Carver, and kidnapped and raped Mrs. Carver.
    \item The Supreme Court held that the Georgia jury's death sentence 
    violated the Eight Amendment, because rape is a crime ``not involving the 
    taking of life.'' In their dissent, Justices Burger and Rehnquist argued 
    that the Eight Amendment does not prohibit states from taking prior 
    behavior into account. While the death penalty may be disproportionate to 
    the current crime, it can act as an effective deterrent.
    \item A related case, \emph{Kennedy v. Louisiana}, involved the rape of a 
    child. The court narrowly upheld that the death penalty was ``grossly 
    disproportionate'' for rape, but Justice Alito issued a scathing dissent 
    questioning the argument that every murder is more ``morally depraved'' 
    than every rape.
\end{enumerate}

\paragraph{Three Strikes: \emph{Ewing v. California}}
~\\\\
Recidivism can justify harsh punishments.

\begin{enumerate}
    \item Ewing stole three golf clubs from a pro shop. With multiple prior 
    felony convictions, California's three strikes law required a minimum 
    sentence of 25 years, which Ewing argued violated the Eighth Amendment.
    \item Justice O'Connor:
    \begin{enumerate}
        \item In \emph{Harmelin v. Michigan}, Justice Kennedy laid out a set 
        of principles for determining proportionality:
        \begin{enumerate}
            \item The primacy of the legislature.
            \item The variety of legitimate penological schemes.
            \item Federalism.
            \item Objectivity.
            \item \textbf{The Eighth Amendment does not require strict 
            proportionality. It only forbids ``grossly disproportionate'' 
            sentences.}
        \end{enumerate}
        \item The court upheld Ewing's 25-year sentence, arguing that 
        ``Ewing's sentence is justified by the State's public-safety interest 
        in incapacitating and deterring recidivist 
        felons...''\footnote{Casebook p.  85.}
    \end{enumerate}
    \item Justice Scalia concurring: the justification for the sentence has 
    nothing to do with proportionality and everything to do with the idea that 
    ``punishment should reasonably pursue the multiple purposes of the 
    criminal law'' (incapacitation, deterrence, retribution, rehabilitation).
    \item Justice Breyer, dissenting: compared two prior cases, \emph{Rummel} 
    and \emph{Solem}, which both involved major prison sentences for 
    recidivist felons who committed relatively small crimes.  In \emph{Solem}, 
    the court found the sentence to be too long, and upheld the sentence in 
    \emph{Rummel}. \emph{Ewing} falls in the ``twilight zone'' between the 
    two. Given that ambiguity, 25 years to life is grossly disproportionate to 
    the crime of shoplifting golf clubs.
\end{enumerate}

\subsection{Statutory Interpretation}

\subsubsection{The Legality Principle}

\begin{enumerate}
    \item \emph{Nullem crimen sine lege, nulla poene sine lege}: ``no crime 
    without law, no punishment without law.'' A person cannot be convicted and 
    punished unless her conduct was defined as criminal.
    \item Three corollaries:
    \begin{enumerate}
        \item Statutes should be understandable to ordinary people.
        \item Statutes should not delegate policy matters on an ad hoc basis.
        \item \textbf{Lenity doctrine}: ambiguous statutes should be 
        interpreted in favor of the accused.
    \end{enumerate}
    \item Rationales:
    \begin{enumerate}
        \item Prevents arbitrary and vindictive application of law.
        \item Enhances individual autonomy by allowing people to act without 
        fear of retroactive punishment.
        \item Provides fair warning so that people can conform their behavior 
        to the law.
    \end{enumerate}
    \item Constitutional foundations:
    \begin{enumerate}
        \item \emph{Ex post facto} clause.\footnote{Art. I, \S 9, cl. 3.}
        \item Bill of attainder clause.\footnote{Art. I, \S\ 9, cl. 3.}
        \item Fifth Amendment.
        \item Fourteenth Amendment.
    \end{enumerate}
    \item \textbf{Institutional competency}: courts should not preempt the 
    legislature's role.
    \item \textbf{Principle of statutory clarity}: a criminal statute must not be ``so 
    vague that men of common intelligence must necessarily guess at its 
    meaning and differ as to its application.''\footnote{\emph{Connally v. 
    Gen. Constr. Co.}, 269 U.S. 385, 391 (1926).} Rationales: same as for the 
    lenity doctrine. Constitutional foundations: Fifth and Fourteenth 
    Amendments.
\end{enumerate}

\subsubsection{Crime without Law: \emph{Commonwealth v. Mochan}}

For a defendant to be convicted and punished, he must have violated a defined 
law. Catch-all offenses like ``injuriously affecting public morality'' can 
suffice.

\begin{enumerate}
    \item The defendant repeatedly made lewd phone calls to a married woman.
    \item His conduct was not prohibited by statute, and no precedential case 
    dealt with the same question.
    \item The trial judge found the defendant guilty of ``intending to debauch 
    and corrupt, and further devising and intending to harass, embarrass, and 
    vilify.''\footnote{Casebook p. 92--93}.
    \item The appellate court held that solicitation of adultery on its own is 
    not indictable, but it affirmed the conviction on the grounds that the 
    defendant's acts ``injuriously affected public morality.''
    \item The dissent raised an \textbf{institutional competency} issue, 
    arguing that the courts should not preempt the legislature's role in 
    defining new crimes.
\end{enumerate}

\subsubsection{Statutory Interpretation and Institutional Competence: 
\emph{Keeler v. Superior Court}}

Institutional competence prevents courts from rewriting statutes in 
interpreting them. If a statute's meaning is ambiguous, the court must look to 
legislative intent and common law meanings.

\begin{enumerate}
    \item A man punched his pregnant ex-wife in the stomach, causing the death 
    of her fetus. He was charged with murder.
    \item The question before the California Supreme Court was whether an 
    unborn fetus was a ``human being'' under the California statutory 
    definition of murder. The court examined common law definitions of 
    murder and concluded that it was intended to protect people who had been 
    ``born alive'' but that it did not protect unborn fetuses.
    \item The prosecution argued that medical advances had shifted the 
    definition of a ``viable'' fetus such that an unborn fetus could be 
    considered a ``human being.'' The court held, however, that such a ruling 
    ``would indeed be to rewrite the statute under the guise of construing 
    it.''\footnote{Casebook p. 99.}
    \item The dissent argued a different interpretation of common law in which 
    a ``quickened'' fetus could be considered a human being.
    \item Soon after this case, the state legislature amended the California 
    murder statute to apply to fetuses.
\end{enumerate}

\subsubsection{Statutory Clarity: \emph{In re Banks}}

Statutes can be found unconstitutional if they are so indefinite as to fail to 
give fair notice and fail to ``define a reasonably ascertainable standard of 
guilt.''

\begin{enumerate}
    \item The defendant was charged under a peeping tom statute that 
    prohibited ``secretly peeping into room occupied by female person.''
    \item The defendant argued the statute was unconstitutional because, if 
    read literally, it would outlaw a wide range of obviously lawful conduct 
    (e.g., looking through a keyhole in your child's bedroom door to make sure 
    she had fallen asleep).
    \item The court reasoned that statutes can be found unconstitutional if 
    they are so indefinite as to fail to give fair notice and fail to ``define 
    a reasonably ascertainable standard of guilt.''\footnote{Casebook p. 105.} 
    In this case, however, the meaning of ``secretly'' is well enough defined 
    to describe invasion of privacy.
\end{enumerate}

\subsubsection{Void for Vagueness: \emph{City of Chicago v. Morales}}

Statutes that fail to clearly define criminal conduct and delineate it from 
lawful conduct are void for vagueness.

\begin{enumerate}
    \item The Chicago city council enacted an ordinance prohibiting ``criminal 
    street gang members'' from ``loitering.''
    \item Justice Stevens: ``...the vagueness that dooms this ordinance is not 
    the product of uncertainty about the normal meaning of `loitering,' but 
    rather about what loitering is covered by the ordinance and what is 
    not.''\footnote{Casebook p. 115.} Under the new statute, ordinary people 
    ``might unwittingly engage in forbidden loitering,'' and law enforcement 
    has too much discretion.
    \item Justice O'Connor, concurring, suggested specific language the 
    legislature might have adopted.
    \item Justice Scalia, dissenting: legislatures are free to regulate 
    against harmless conduct, and in doing so they do not violate the 
    constitution.
\end{enumerate}

\subsubsection{Interpreting Statutory Language: \emph{Muscarello v. United States}}

A statute's meaning can turn on a single word. Courts are charged with 
interpreting statutory language.

\begin{enumerate}
    \item The question was whether the term ``carry'' in a firearm statute 
    included carrying in a car, or whether it was restricted to carrying on 
    the person.
    \item Justice Breyer traced the definition of the word in a range of 
    contexts, including a brief empirical reading of news stories, to argue 
    that it includes carrying in a car.
    \item Justice Ginsburg dissented with her 
    own range of examples pointing in the opposite direction, and argued that 
    the lenity principle should resolve the ambiguity in favor of the 
    defendant.
\end{enumerate}

